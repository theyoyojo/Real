\documentclass{article}

\usepackage{amsmath, amssymb, amsthm}
\newcommand{\reals}{\ensuremath{\mathbb{R}}}
\newcommand{\rats}{\ensuremath{\mathbb{Q}}}
\newcommand{\irats}{\ensuremath{\reals \backslash \rats}}
\newcommand{\nats}{\ensuremath{\mathbb{N}}}
\newcommand{\eps}{\ensuremath{\epsilon}}
\newcommand{\infx}[1]{\ensuremath{\underset{#1}{\inf}}}
\newcommand{\supx}[1]{\ensuremath{\underset{#1}{\sup}}}
\newcommand{\pt}[1]{\textrm{ #1 }}
\newtheorem{clm}{Claim}

\title{Real Analysis Assignment 12}
\author{Joel Savitz}

\begin{document}

\maketitle

\textbf{Problem 1:}

Define $f(x) = \begin{cases}
	-3, & x = 4 \\
	4, & x = 5 \\
	0, & x \in [3,6], x \neq 4, x \neq 5 \\
\end{cases}$ on the interval $[3,6]$.

\begin{clm}
	$f$ is Riemann integrable over $[3,6]$ and $\underset{[3,6]}{\int}f = 0$.
	% $f$ is Riemann integrable over $[3,6]$ and $\int_3^6 f = 0$.
\end{clm}

\begin{proof}
	Let $\delta > 0$ be small,
	and define the partition
	$P = \{ 3, 4-\delta, 4+\delta, 5-\delta, 5+\delta, 6 \}$.
	We calculate the lower sum:
	\begin{align}
		L(f,P)   &= \sum_{j=1}^5 \infx{x \in I_j} f(x) \cdot \Delta x_j \\
			 & = 0(4-\delta - 3) -3(4+\delta-4+\delta) + 0(5-\delta-4-\delta) \nonumber \\
			 &+ 0(5+\delta-5+\delta) + 0(6 - 5 - \delta) \\
			 &= -6\delta
	\end{align}
	And the upper sum:
	\begin{align}
		U(f,P)   &= \sum_{j=1}^5 \supx{x \in I_j} f(x) \cdot \Delta x_j \\
			 & = 0(4-\delta - 3) + 0(4+\delta-4+\delta) + 0(5-\delta-4-\delta) \nonumber \\
			 &+ 4(5+\delta-5+\delta) + 0(6 - 5 - \delta) \\
			 &= 8\delta
	\end{align}
	Let $\eps > 0$ and impose the constraint that $\delta < \frac{\eps}{14}$.
	Then, $U(f,P) - L(f,P) = 8\delta - (-6\delta) = 14\delta < 14\frac{\eps}{14} = \eps$.
	Therefore,
	$\forall \eps > 0,
	\exists P_\eps = P \in \mathcal{P}([3,6]),
	U(f,P_\eps) - L(P,\eps) < \eps$,
	so $f$ satisfies Riemann's condition
	on the interval $[3,6]$.
	This is true if and only if
	$f$ is Riemann integrable over $[3,6]$.
	Then, since $L(f,P) \le L(f) \le U(f) \le U(f,P)$,
	we have for a small $\delta > 0$
	that $-6\delta \le L(f) \le U(f) \le 8\delta$,
	and since when $\delta$ is  arbitrarily small,
	$-6\delta$ and $8\delta$ will be arbitrarily close to $0$,
	so $0 \le L(f) \le U(f) \le 0$,
	implying that $L(f) = U(f) = 0$
	and equivalently that 
	$\underset{[3,6]}{\int}f = 0$.
\end{proof}

\textbf{Problem 2:}

Define $f(x) = \begin{cases}
	3, & 1 \le x < 2 \\
	2, & 2 \le x \le 4 \\
	4, & 4 < x \le 5 \\
\end{cases}$ on the interval $[1,5]$.

\begin{clm}
	$f$ is Riemann integrable on $[1,5]$ and $\underset{[1,5]}{\int}f = 11$.
\end{clm}

\begin{proof}
	By inspection, $\forall x \in [1,5], |f(x)| \le 4$,
	therefore $f$ is bounded on $[1,5]$.
	Define the finite sequence of intervals
	$a_1 = (2-\frac{\eps}{8},2+\frac{\eps}{8})$
	and $a_2 = (4-\frac{\eps}{8}, 4+\frac{\eps}{8})$.
	Then, the set of discontinuities of $f$ in $[1,5]$,
	i.e. $\{2,4\}$, is a subset of $a_1 \cup a_2$.
	Furthermore, $(2+\frac{\eps}{8} - (2-\frac{\eps}{8})) + 
	(4+\frac{\eps}{8} - (4-\frac{\eps}{8})) = \frac{\eps}{2} < \eps$,
	therefore the set of discontinuities of $f$ on $[1,5]$
	has Lebesgue measure $0$ and is bounded on that same interval.
	This is true if and only if
	$f$ is Riemann integrable on $[1,5]$.
	
	Let $\delta > 0$ be small
	and define the partition
	$P = \{1, 2-\delta, 2+\delta,4-\delta,4+\delta,5\}$.
	We calculate the lower sum:
	\begin{align}
		L(f,P) & =\sum_{j=1}^5 \infx{x \in I_j} f(x) \cdot \Delta x_j \\
		       & = 3(2-\delta-1) + 2(2+\delta - 2+\delta) + 2(4-\delta-2-\delta) \nonumber \\
		       & + 2(4 + \delta - 4 + \delta) + 4(5-4-\delta) \\
		       & = 11-3\delta
	\end{align}
	And the upper sum:
	\begin{align}
		U(f,P) & =\sum_{j=1}^5 \supx{x \in I_j} f(x) \cdot \Delta x_j \\
		       & = 3(2-\delta-1) + 3(2+\delta - 2+\delta) + 2(4-\delta-2-\delta) \nonumber \\
		       & + 4(4 + \delta - 4 + \delta) + 4(5-4-\delta) \\
		       & = 11+3\delta
	\end{align}
	As in the previous problem,
	we have $L(f,P) \le L(f) \le U(f) \le U(f,P)$,
	so for an arbitrarily small $\delta > 0$, we have
	$11-3\delta \le L(f) \le U(f) \le 11 + 3\delta$
	therefore $11 = L(f) = U(f)$,
	or equivalently
	$\underset{[1,5]}{\int}f = 11$.
\end{proof}

\end{document}
