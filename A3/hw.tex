\documentclass{article}

\usepackage{amsmath, amssymb, amsthm}
\newcommand{\reals}{\ensuremath{\mathbb{R}}}
\newcommand{\nats}{\ensuremath{\mathbb{N}}}
\newcommand{\eps}{\ensuremath{\epsilon}}
\newcommand{\neps}{\ensuremath{N_\epsilon}}
\newcommand{\overn}[1]{\ensuremath{\frac{#1}{n}}}
\newcommand{\movern}{\overn{(-1)^n}}
\newcommand{\bsn}{\sqrt{9-\movern}}
\newcommand{\csn}{\sqrt{4n^2+9n+1}}
\newtheorem{clm}{Claim}

\title{Real Analysis Assignment 3}
\author{Joel Savitz}

\begin{document}

\maketitle

\textbf{Problem 1:}

\begin{clm} \label{c1}
	$\lim \frac{6n+5}{4n+7} = \frac{3}{2}$
\end{clm}

\begin{proof}
	Let $\eps \in \reals : \eps > 0$.
	Then, since \reals\ is closed under the field operations,
	we have $\frac{1}{8}(\frac{11}{\eps} - 14) \in \reals$.
	Since $(1 \in \reals \land 1 > 0) \land \frac{1}{8}(\frac{11}{\eps} - 14) \in \reals$,
	by the Archimedian property of \reals\,
	we have $\exists n \in \nats: 1 \cdot n = n > \frac{1}{8}(\frac{11}{\eps} - 14)$.
	This inequality implies that $8n + 14 > \frac{11}{\eps} \implies \frac{11}{8n+14} < \eps$.
	Since $x = |-x|$, we have $|\frac{-11}{8n+14}| < \eps$.
	Therefore, $|\frac{12n + 10 - 12n - 21}{8n+14}| = |\frac{2(n+5) - 3(4n+7)}{8n+14}| =
	|\frac{6n+5}{4n+7} - \frac{3}{2}| < \eps$.
	Thus we have demonstated that $\forall \eps \in \reals : \eps > 0, \exists \neps \in \nats:
	\forall n \in \nats: n > \neps, |\frac{6n+5}{4n+7} - \frac{3}{2}| < \eps$,
	which is exactly the definition that
	$\lim \frac{6n+5}{4n+7} = \frac{3}{2}$.
\end{proof}

\textbf{Problem 2:}

\begin{clm} \label{c2}
	$\lim \sqrt{9-\frac{(-1)^n}{n}} = 3$
\end{clm}

\begin{proof}
	Let $\eps \in \reals: \eps > 0$.
	By the Archimedian property of \reals,
	$\exists n \in \nats: n > \frac{1}{3\eps}$.
	Then, $\frac{1}{3n} = \frac{\overn{1}}{3}< \eps$.
	Since $\forall x \in \reals, \exists y \in \reals: \sqrt{y}=x \implies x > 0$,
	we have $\bsn > 0 \implies \bsn + 3 > 3 \implies \frac{1}{\bsn + 3} < \frac{1}{3}
	\implies \frac{\overn{1}}{\bsn + 3} < \frac{\overn{1}}{3}$.
	Any natural number is either even or odd, i.e.
	$\forall n \in \nats, \exists k \in \nats: (n = 2k) \lor (n = 2k-1)$.
	Then, $n = 2k \implies \movern = \overn{1} \land n = 2k-1 \implies \movern = \overn{-1}$,
	so we have $\forall n \in \nats, \movern \le \overn{1}$.
	Thus, $\frac{\movern}{\bsn + 3} \le \frac{\overn{1}}{\bsn + 3}$.
	\textbf{NO THIS IS FALSE}
	Since $x = |-x|$, we have $\frac{\movern}{\bsn + 3} = |\frac{-\movern}{\bsn + 3}|$,
	and then $|\frac{-\movern}{\bsn + 3}| = |\frac{9-\movern-9}{\bsn + 3}| =
	|\frac{(\bsn)^2-3^2}{\bsn + 3}| = |\bsn - 3|$.
	By transitivity, $|\bsn - 3| \le \frac{\overn{1}}{\bsn + 3} < \frac{1}{3n} < \eps
	\implies |\bsn - 3| < \eps$.
	Thus we have demonstated that $\forall \eps \in \reals : \eps > 0, \exists \neps \in \nats:
	\forall n \in \nats: n > \neps,
	|\bsn - 3| < \eps$.
	which is exactly the definition that
	$\lim \sqrt{9-\frac{(-1)^n}{n}} = 3$.
\end{proof}

\textbf{Problem 3:}

\begin{clm} \label{c3}
	$\lim \frac{n}{\csn} = \frac{1}{2}$
\end{clm}

\begin{proof}
	Let $\eps \in \reals: \eps > 0$.
	By the Archimedian property of \reals,
	$\exists n \in \nats: n > \frac{1}{2}(\frac{1}{\eps - \frac{9}{2}})$.
	Then, $\frac{1}{2n} + \frac{9}{2} = \frac{1}{2n} + \frac{9n}{2n} = \frac{9n+1}{2n}< \eps$.
	Since $\forall n \in \nats, 1 < 1 + 4n^2 + 9n$, we have $1 < \csn$.
	Then, $2n + \csn > 2n \implies \frac{1}{2n + \csn} < \frac{1}{2n}
	\implies \frac{9n+1}{2n + \csn} < \frac{9n+1}{2n} $. 
	Furthermore, since $\csn > 1 \implies 2\csn > 1$,
	% fix formatting
	we have $(2n + \csn)(2\csn) > 2n + \csn \implies \frac{1}{(2n + \csn)(2\csn)} < \frac{1}{2n + \csn}
	\implies \frac{9n+1}{(2n + \csn)(2\csn)} < \frac{9n+1}{2n + \csn}$.
	Since $\forall x \in \reals: x > 0, |-x| = x$,
	we have $\frac{9n+1}{(2n + \csn)(2\csn)} = |\frac{-(9n+1)}{(2n + \csn)(2\csn)}|
	= |\frac{4n^2 -(4n^2 +9n+1)}{(2n + \csn)(2\csn)}| = |\frac{(2n)^2 - (\csn)^2}{(2n + \csn)(2\csn)}|
	= |\frac{2n - \csn}{2\csn}| = |\frac{n}{\csn} - \frac{1}{2}|$.
	Then by transitivity, $|\frac{n}{\csn} - \frac{1}{2}| < \frac{9n+1}{2n + \csn} < \frac{9n+1}{2n} < \eps
	\implies |\frac{n}{\csn} - \frac{1}{2}| < \eps$
	Thus we have demonstated that $\forall \eps \in \reals : \eps > 0, \exists \neps \in \nats:
	\forall n \in \nats: n > \neps,
	|\frac{n}{\csn} - \frac{1}{2}| < \eps$
	which is exactly the definition that
	$\lim \frac{n}{\csn} = \frac{1}{2}$.
\end{proof}

\end{document}
