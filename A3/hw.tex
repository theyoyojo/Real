\documentclass{article}

\usepackage{amsmath, amssymb, amsthm}
\newcommand{\reals}{\ensuremath{\mathbb{R}}}
\newcommand{\nats}{\ensuremath{\mathbb{N}}}
\newcommand{\eps}{\ensuremath{\epsilon}}
\newcommand{\neps}{\ensuremath{N_\epsilon}}
\newcommand{\overn}[1]{\ensuremath{\frac{#1}{n}}}
\newcommand{\movern}{\overn{(-1)^n}}
\newcommand{\bsn}{\sqrt{9-\movern}}
\newcommand{\csn}{\sqrt{4n^2+9n+1}}
\newcommand{\dsn}{\sqrt{n^2 + 5n}}
\newtheorem{clm}{Claim}

\title{Real Analysis Assignment 3}
\author{Joel Savitz}

\begin{document}

\maketitle

\textbf{Problem 1:}

\begin{clm} \label{c1}
	$\lim \frac{6n+5}{4n+7} = \frac{3}{2}$
\end{clm}

\begin{proof}
	Let $\eps \in \reals : \eps > 0$.
	By the Archimedian property of \reals\,
	we have $\exists N_\eps \in \nats: N_\eps > \frac{1}{8}(\frac{11}{\eps} - 14)$.
	Then, $\forall n > N_\eps, n > \frac{1}{8}(\frac{11}{\eps})$.
	This inequality implies the following: 
	\begin{align}
		\frac{11}{8n+14} < \eps \\
		\forall x < 0, |x| = -x \implies \frac{11}{8n+14} = |\frac{-11}{8n+14}| < \eps \\
	|\frac{-11}{8n+14}| = |\frac{12n + 10 - 12n - 21}{8n+14}| = |\frac{2(6n+5) - 3(4n+7)}{8n+14}| \\
	|\frac{2(6n+5) - 3(4n+7)}{8n+14}| = |\frac{6n+5}{4n+7} - \frac{3}{2}| < \eps
	\end{align}
	Thus we have demonstated that $\forall \eps \in \reals : \eps > 0, \exists \neps \in \nats:
	\forall n \in \nats: n > \neps, |\frac{6n+5}{4n+7} - \frac{3}{2}| < \eps$,
	which is exactly the definition that
	$\lim \frac{6n+5}{4n+7} = \frac{3}{2}$.
\end{proof}

\textbf{Problem 2:}

\begin{clm} \label{c2}
	$\lim \sqrt{9-\frac{(-1)^n}{n}} = 3$
\end{clm}

\begin{proof}
	Let $\eps \in \reals: \eps > 0$.
	By the Archimedian property of \reals,
	$\exists N_\eps \in \nats: N_\eps > \frac{1}{3\eps}$,
	so $\forall n > N_\eps, n > \frac{1}{3\eps}$.

	Assume for contradiction that
	$\frac{(-1)^n}{n} > 1$.
	Then, 
	$(\frac{(-1)^n}{n})^2 = \frac{1}{n^2} > 1$,
	so $n^2 < 1$.
	Since $n > 0$, this implies that $n < 1$,
	but of course $n \geq 1$,
	which is a contradiciton,
	therefore
	$\frac{(-1)^n}{n} < 1 < 9$,
	so $9 - \frac{(-1)^n}{n} > 0$,
	and $\bsn$ exists.

	Then, since $\bsn > 0$, we have:

	\begin{align}
		\bsn + 3 > 3 \\
		\frac{1}{\bsn + 3} < \frac{1}{3} \\
	\frac{\overn{1}}{\bsn + 3} = | \frac{\overn{-(-1)^n}}{\bsn + 3}| < \frac 1{3n} \\
	|\frac{-\movern}{\bsn + 3}| = |\frac{9-\movern-9}{\bsn + 3}| \\
|\frac{9-\movern-9}{\bsn + 3}| = |\frac{(\bsn)^2-3^2}{\bsn + 3}| = |\bsn - 3|.
\end{align}

	By transitivity, $|\bsn - 3| < \frac{1}{3n} < \eps
	\implies |\bsn - 3| < \eps$.
	Thus we have demonstated that $\forall \eps \in \reals : \eps > 0, \exists \neps \in \nats:
	\forall n \in \nats: n > \neps,
	|\bsn - 3| < \eps$.
	which is exactly the definition that
	$\lim \sqrt{9-\frac{(-1)^n}{n}} = 3$.
\end{proof}

\textbf{Problem 3:}

\begin{clm} \label{c3}
	$\lim \frac{n}{\csn} = \frac{1}{2}$
\end{clm}

\begin{proof}
	Let $\eps \in \reals: \eps > 0$.
	By the Archimedian property of \reals,
	$\exists N_\eps: N_\eps > \frac{18}{8\eps}$.
	Then, $\forall n > N_\eps, n > \frac{18}{8\eps}$.
	We can then derive the following inequalities and equations:
	\begin{align}
		n > \frac{18}{8\eps} \implies \frac{18}{8n} < \eps \\
	      \frac{ 9n + 9n }{8n^2 + 18n} = \frac{18}{8n + 18} < \frac{18}{8n} \\
	      \frac{ 9n + 1 }{2(4n^2 + 9n + 1)} =
	      \frac{ 9n + 1 }{8n^2 + 18n + 2} <
				\frac{ 9n + 9n }{8n^2 + 18n + 2} <
				\frac{ 9n + 9n }{8n^2 + 18n} \\
				\frac{ 9n + 1 }{(\csn)(2\csn)} =
	      \frac{ 9n + 1 }{2(4n^2 + 9n + 1)} \\
	      \frac{ (4n^2 + 9n + 1) -4n^2}{(2n + \csn)(2\csn)} <
				\frac{ 9n + 1 }{(\csn)(2\csn)} \\
				\frac{ (4n^2 + 9n + 1) -4n^2}{(2n + \csn)(2\csn)} =
	      \frac{\csn -2n}{(2\csn)} \\
	      \frac{\csn -2n}{(2\csn)} = \frac{-(2n - \csn)}{2\csn}
	\end{align}

	Now assume for a contradiction that $\csn \le 2n$.
	This inequality implies:
	\begin{align}
		4n^2 + 9n + 1 \le & 4n^2 \\
		9n + 1 \le & 0 \\
		n \le & \frac{-1}{9} < 0
	\end{align}
	But $n > 0$ since $n \in \nats$,
	which is a contradiction,
	therefore $\csn > 2n \implies 0 > 2n - \csn$,
	so $|
	\frac{2n - \csn}{2\csn}
	| =
	\frac{-(2n - \csn)}{2\csn}$.
	Finally,
	$|\frac{2n - \csn}{2\csn}| = |\frac{n}{\csn} - \frac{1}{2}|$,
	and by the transitivity of the above order relations,
	we have $|\frac{n}{\csn} - \frac{1}{2}| < \eps$. 
	Thus we have demonstated that $\forall \eps \in \reals : \eps > 0, \exists \neps \in \nats:
	\forall n \in \nats: n > \neps,
	|\frac{n}{\csn} - \frac{1}{2}| < \eps$
	which is exactly the definition that
	$\lim \frac{n}{\csn} = \frac{1}{2}$.
\end{proof}

\textbf{Problem 4:}

\begin{clm} \label{c4}
	$\lim(\dsn - n) = \frac{5}{2}$
\end{clm}

\begin{proof}
	Let $\eps \in \reals: \eps > 0$.
	By the Archimedian property of \reals,
	$\exists N_\eps \in \nats: N_\eps > \frac{25}{4\eps}$.
	Then, $\forall n > N_\eps, n > \frac{25}{4\eps}$.
	We have the following inequalities and equations:
	\begin{align}
		n > \frac{25}{4\eps} \implies & \frac{25}{4n} < \eps \\
		\frac{5}{2} + \sqrt{n^2 + 5n} > 0 \implies &
		\frac{\frac{25}{4}}{n + \frac{5}{2} + \sqrt{n^2 + 5n}} < \frac{25}{4n} \\
		   & \frac{\frac{25}{4}}{n + \frac{5}{2} + \sqrt{n^2 + 5n}} =
	   \frac{(n^2 + 5n) + \frac{25}{4} -(n^2 + 5n)}{n + \frac{5}{2} + \sqrt{n^2 + 5n}} \\
		(a-b) = \frac{a^2-b^2}{a + b} \implies &
	   \frac{(n^2 + 5n) + \frac{25}{4} -(n^2 + 5n)}{n + \frac{5}{2} + \sqrt{n^2 + 5n}} =
	   n + \frac{5}{2} - \sqrt{n^2 + 5n} \\
		n + \frac{5}{2} - \sqrt{n^2 + 5n} = &
		- (\sqrt{n^2 + 5n} - (n + \frac{5}{2})
	\end{align}

	Now, assume for the purpose of contradiction that
	$n + \frac{5}{2} \le \sqrt{n^2 + 5n}$. Then, we have:
	\begin{align}
		n^2 + 5n + \frac{25}{4} \le & n^2 + 5n \\
		\frac{25}{4} \le & 0
	\end{align}
	But of course $\frac{25}{4} > 0$, which is a contradiction.
	Thus,
	$n + \frac{5}{2} > \sqrt{n^2 + 5n}$.
	So $- (\sqrt{n^2 + 5n} - (n + \frac{5}{2}))
	=|\sqrt{n^2 + 5n} - (n + \frac{5}{2})|
	=|\sqrt{n^2 + 5n} - n - \frac{5}{2})|$,
	and by transitivity of the above order relations,
	we have 
	$|\sqrt{n^2 + 5n} - n - \frac{5}{2})| < \eps$.
	Thus we have demonstated that $\forall \eps \in \reals : \eps > 0, \exists \neps \in \nats:
	\forall n \in \nats: n > \neps,
	|\dsn - n - \frac{5}{2}| < \eps$,
	which is exactly the definition that
	$\lim(\dsn - n) = \frac{5}{2}$.
\end{proof}

\textbf{Problem 5:}

\begin{clm} \label{clm}
	$\lim(3 + 2(-1)^n) \neq 5$
\end{clm}

\begin{proof}
	Let $\eps = 1$, let $N \in \nats$, and let $m = 2N + 1$.
	Then, $|3 - 2(-1)^m - 5| = |3 - 2(-1)^2(-1) - 5| =
	|-4| = 4 \geq 1 = \eps$.
	Thus we have demonstrated that $\exists \eps \in \reals : \eps > 0 : \forall N_\eps \in \nats,
	\exists n \in \nats: n > N_\eps: |3 - 2(-1)^n - 5| \geq \eps
	\iff \neg(\forall \eps \in \reals : \eps > 0, \exists \neps \in \nats:
	\forall n \in \nats: n > \neps, | 3 -2(-1)^n - 5| < \eps)$,
	which is the negation of
	$\lim(3 + 2(-1)^n) = 5$,
	and
	$\neg(\lim(3 + 2(-1)^n) = 5) \iff \lim(3 + 2(-1)^n) \neq 5$.

\end{proof}

\end{document}
