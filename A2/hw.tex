\documentclass{article}

\usepackage{amsmath, amssymb, amsthm}
\newcommand{\reals}{\mathbb{R}}
\newcommand{\nats}{\mathbb{N}}
\newtheorem{thm}{Theorem}

\title{Real Analysis Assignment 2}
\author{Joel Savitz}

\begin{document}

\maketitle

Note: For the scope of this document, let $\ni$ denote ``such that''

\textbf{1:}

Suppose $E$ is a non-empty subset of $\reals$ that is bounded below,
i.e. $E \subseteq \reals \ni (E \neq \emptyset) \land (\exists x \in \reals \ni \forall y \in E, x \le y)$.

\begin{thm} \label{thm1}
	$\inf E$ exists.
\end{thm}

\begin{proof}
	Let $A$ be the set of all lower bounds of $E$,
	and let $B$ be the set of all non-lower bounds of $E$.
	More precisely,
	let $A = \{ x \in \reals: \forall y \in E, x \le y \}$,
	and let $B = \{x \in \reals: \exists y \in E: x > y \}$.

	Suppose $x \in \reals$. Then,
	we have either $\exists y \in \reals \ni x > y$
	or we have $\neg (\exists y \in \reals \ni x > y) = \forall y \in \reals x \le y$,
	i.e. either $x$ is not a lower bound of $E$ or it is.
	If the former, then by definition $x \in B$,
	and if the latter, then $x \in A$.
	Since this covers every possible element of $\reals$,
	we have that $A \cup B = \reals$.

	Suppose $x \in E$.
	Then, since $x < x + 1$,
	we have that $\exists y \in E \ni y < x + 1$,
	and this $y$ is our previously introduced $x$,
	therefore $x + 1 \in B$ since it is not a lower bound of $E$,
	and so $B \neq \emptyset$.

	Since $E$ is bounded below,
	we have that $\exists x \in \reals \ni \forall y \in E, x \le y$
	by the definition of a lower bound,
	and this is exactly the definition of membership in $A$, so $x \in A$,
	and therefore $A \neq \emptyset$.

	Now let $a \in A$ and $b \in B$.
	$b \in B \iff \exists y \in E \ni y < b$,
	so let $\beta \in E$ be such that $\beta < b$.
	$a \in A \iff \forall x \in E, a \le x$,
	and since $\beta \in E$, we have $a \le \beta$.
	Putting these together, we have $a \le \beta < b$,
	and by transitivity of order we have $a < b$,
	therefore in general we have that
	$\forall a \in A, \forall b \in B, a < b$.

	Since $A \cup B = \reals \land A \neq \emptyset \land B \neq \emptyset \land
	\forall a \in A, b \in B, a < b$,
	we have that $(A,B)$ is a Dedekind cut of $\reals$.

	By Dedekind's axiom,
	we have that for any Dedekind cut $(X,Y)$ of $\reals$,
	there exists some $z \in \reals \ni \forall x \in X, \forall y \in Y, x \le z \le y$.

	Therefore, there exists some $c \in \reals$
	such that $\forall a \in A, \forall b \in B, a \le c \le b$.

	Assume that $c$ is not a lower bound of $E$.
	Then, $\exists x \in E \ni x < c$.
	Let $y = \frac{x + c}{2}$.
	Then, $x < y < c$.
	Since $\exists x \in E \ni x < y$,
	we have $y \in B$ by definition,
	so $\exists b \in B \ni b < c$,
	however we have by Dedekind's axiom
	that $\forall b \in B, c \le B$,
	which is a contradiction.
	Therefore $c$ must be a lower bound of $E$.

	Let $x$ be some lower bound of $E$.
	Then, $\forall y \in E, x \le y$, so $x \in A$.
	By Dedekind's axiom, we have that
	$\forall a \in A, a \le c$,
	therefore $x \le c$,
	so $c$ must be the greatest lower bound of $E$,
	or in other words, $c = \inf E$.

	Since we have constructed $\inf E$,
	we have demonstrated that $\inf E$ exists,
	and this proves theorem \ref{thm1}.

\end{proof}

\medskip
\textbf{2:}

Suppose $E_1, E_2$ are nonempty subsets of $\reals$ bounded above, where $E_1$ is a subset of $E_2$,
i.e. $E_1 \subseteq \reals, E_2 \subseteq \reals \ni E_1 \subseteq E_2 \land E_1 \neq \emptyset \neq E_2
\land (\exists x \in \reals \ni \forall y \in E_1, y \le x)
\land (\exists x \in \reals \ni \forall y \in E_2, y \le x)$.

\begin{thm} \label{thm2}
	$\sup E_1 \le \sup E_2$.
\end{thm}

\begin{proof}
	By the Dedekind property of $\reals$,
	$\sup E_2$ exists.
	Let $\beta = \sup E_2$.
	Then, $\forall x \in E_2, x \le \beta$.

	Let $x \in E_1$.
	Since $E_1 \subseteq E_2 \iff y \in E_1 \implies y \in E_2$,
	we have $x \in E_2$,
	therefore $x \le \beta$,
	so $\forall x \in E_1, x \le \beta$,
	and we have that $\beta$ is an upper bound of $E_1$.

	By the Dedekind property of $\reals$,
	$\sup E_1$ exists.
	Let $\alpha = \sup E_1$.
	By definition, $\alpha$ is the least upper bound of $E_1$,
	i.e. $(\forall x \in \reals \ni \forall y \in E_1, x \geq y)(\alpha \le x)$.
	Since $\beta$ an upper bound of $E_1$, i.e. $\forall x \in E_1, \beta \geq x$,
	we must then have that $\alpha \le \beta$,
	or in other words $\sup E_1 \le \sup E_2$.
	This proves theorem \ref{thm2}.
\end{proof}

\textbf{3:}

Let $A = \{ x \in \reals: x = 2 + \frac{1}{n} \textrm{ for some } n \in \nats\}$.

\begin{thm} \label{thm3}
	$\inf A = 2$
\end{thm}

\begin{proof}
	Let $n \in \nats$.
	Then, $n > 0 \implies \frac{1}{n} > 0$,
	thus $2 + \frac{1}{n} > 2 + 0 = 2$.
	so $\forall n \in \nats, 2 + \frac{1}{n} > 2$,
	therefore $2$ must be a lower bound of $A$.

	Let $\epsilon \in \reals \ni \epsilon > 0$.
	By the archimedian property of $\reals$,
	$\exists n \in \nats \ni \epsilon > \frac{1}{n} > 0$,
	therefore $2 + \epsilon > 2 + \frac{1}{n} > 2$.
	$2 + \frac{1}{n} \in A$ by definition,
	therefore $\exists x \in A \ni x < 2 + \epsilon$
	for all $\epsilon > 0$,
	where $x = 2 + \frac{1}{n}$.
	Since any arbitrarily small amount greater than $2$
	is not a lower bound of $A$,
	and since $2$ is a lower bound of $A$,
	$2$ must therefore be the greatest lower bound of $A$,
	or in other words, $2 = \inf A$.
	This proves theorem \ref{thm3}.

\end{proof}

\end{document}
