\documentclass{article}

\usepackage{amsmath, amssymb}
\newcommand{\reals}{\mathbb{R}}

\title{Real Analysis Assignment 1}
\author{Joel Savitz}

\begin{document}

\maketitle

\textbf{1(a):}

Suppose we have the following equation:

\begin{align} \label{eq1}
	|x-1| + |x+2| = 0.5
\end{align}

We have $x \in \reals $ if and only if:
\begin{align} \label{eq0}
	x \le -2 \lor -2 \le x \le 1 \lor 1 \le x
\end{align}

since $(-\infty,-2] \cup [-2,1] \cup [1,\infty) = \reals$.

Assume $x \le -2$.
Then, $x-1 < 0 \implies |x-1| = -(x-1)$,
and $x+2 \le 0 \implies |x+2| = -(x+2)$,
so we can solve (\ref{eq1}) as follows:

\begin{align}
	& -(x-1) -(x+2) & = 0.5 \\
	= & -x+1 -x-2 & = 0.5 \\
	= & -2x-1 & = 0.5 \\
	\implies & -2x & = 1.5 \\
	\label{eq2} \implies & x & = \frac{1.5}{-2} = \frac{-3}{4}
\end{align}

However, we have by assumption that $x \le -2$,
and by (\ref{eq2}) we have that $x = \frac{-3}{4} > -2$,
which is a contradiction.
Therefore, $x \le -2$ is false for any real solution to (\ref{eq1}).

Assume $-2 \le x \le 1$.
Then, $x-1 \le 0 \implies |x-1| = -(x-1)$,
and $x+2 \geq 0 \implies |x+2| = x+2$,
so we can solve (\ref{eq1}) as follows:

\begin{align}
	& -(x-1) + x+2 & = 0.5 \\
	= & -x+1 + x+2 & = 0.5 \\
	= & 3 & = 0.5
\end{align}

But of course, $3 = 0.5$ is absurd,
therefore we reach a contradiction
and conclude that $-2 \le x \le 1$ is false for any real solution to (\ref{eq1}).

Assume $x \geq 1$.
Then, $x-1 \geq 0 \implies |x-1| = x-1$,
and $x+2 > 0 \implies |x+2| = x+2$,
so we can solve (\ref{eq1}) as follows:

\begin{align}
	& x-1 +x+2 & = 0.5 \\
	= & 2x+1 & = 0.5 \\
	\implies & 2x & = -0.5 \\
	\label{eq3} \implies & x & = \frac{-0.5}{2} = \frac{-1}{4}
\end{align}

However, we have by assumption that $x \geq 1$,
and by (\ref{eq3}) we have that $x = \frac{-1}{4} < 1$,
which is a contradiction.
Therefore, $x \geq 1$ is false for any real solution to (\ref{eq1}).

Finally, since we have $x \not \in (-\infty,-2] \land x \not\in [-2,1] \land x \not \in [1,\infty)$,
we have $x \not \in \reals = (-\infty,-2] \cup [-2,1] \cup [1,\infty)$,
therefore we have demonstrated that there are no real solutions to (\ref{eq1}).

\medskip
\textbf{1(b):}

Suppose we have the following equation:

\begin{align} \label{eq4}
	|x-1| + |x+2| = 3.5
\end{align}

We have $x \in \reals $ if and only if (\ref{eq0}) holds
since $(-\infty,-2] \cup [-2,1] \cup [1,\infty) = \reals$.

Assume $x \le -2$.
Then, $x-1 < 0 \implies |x-1| = -(x-1)$,
and $x+2 \le 0 \implies |x+2| = -(x+2)$,
so we can solve (\ref{eq4}) as follows:

\begin{align}
	& -(x-1) -(x+2) & = 3.5 \\
	= & -x+1 -x-2 & = 3.5 \\
	= & -2x-1 & = 3.5 \\
	\implies & -2x & = 4.5 \\
	\label{eq5} \implies & x & = \frac{4.5}{-2} = \frac{-9}{4}
\end{align}

Indeed, $x = \frac{-9}{4} \le -2$ so this solution is consistent
with our constraints and it must be a member of the solution set for (\ref{eq4}).

Assume $-2 \le x \le 1$.
Then, $x-1 \le 0 \implies |x-1| = -(x-1)$,
and $x+2 \geq 0 \implies |x+2| = x+2$,
so we can solve (\ref{eq4}) as follows:

\begin{align}
	& -(x-1) + x+2 & = 3.5 \\
	= & -x+1 + x+2 & = 3.5 \\
	= & 3 & = 3.5
\end{align}

But of course, $3 = 3.5$ is absurd,
therefore we reach a contradiction
and conclude that $-2 \le x \le 1$ is false for any real solution to (\ref{eq4}).

Assume $x \geq 1$.
Then, $x-1 \geq 0 \implies |x-1| = x-1$,
and $x+2 > 0 \implies |x+2| = x+2$,
so we can solve (\ref{eq4}) as follows:

\begin{align}
	& x-1 +x+2 & = 3.5 \\
	= & 2x+1 & = 3.5 \\
	\implies & 2x & = 2.5 \\
	\label{eq6} \implies & x & = \frac{2.5}{2} = \frac{5}{4}
\end{align}

Indeed, $x = \frac{5}{4} \geq 1$ so this solution is consistent
with our constraints and it must be a member of the solution set for (\ref{eq4}).

Finally, we can describe the real solutions
to (\ref{eq4}) by $x \in \{ \frac{-9}{4}, \frac{5}{4} \}$.

\medskip
\textbf{1(c):}

Suppose we have the following inequality: 

\begin{align} \label{eq7}
	|x-2| < 3
\end{align}

Some $x \in \reals$ solves (\ref{eq7}) if and only if
$x \in (-\infty,2] \lor x \in [2, \infty)$
since $(-\infty,2] \cup [2,\infty) = \reals$.


Assume $x \le 2$.
Then, $x - 2 \le 0 \implies |x-2| = -(x-2) = -x+2$,
so we can solve (\ref{eq7}):

\begin{align}
	&-x+2 &< &3 \\
	\implies &-x &< &1 \\
	\implies &x & > &-1
\end{align}

Therefore we have $-1 < x \le 2$ when $x \le 2$

Assume $x \geq 2$.
Then, $x - 2 \geq 0 \implies |x-2| = x-2$
so we can solve (\ref{eq7}):

\begin{align}
	&x-2 &< 3 \\
	\implies &x &< 5
\end{align}

Therefore we have $2 \le x < 5$ when $x \geq 2$

Since those two cases describe every case of $x \in \reals$,
we must have that $-1 < x \le 2 \lor 2 \le x < 5$,
thus we have demonstrated that $-1 < x < 5$ solves (\ref{eq7}).

\medskip
\textbf{1(d):}

Suppose we have the following inequality:

\begin{align} \label{eq8}
	x + \frac{2-4x}{x+1} > 0
\end{align}

Since $x = \frac{x(x+1)}{x+1}$,
we can simplify (\ref{eq8}) as follows:

\begin{align}
	\frac{x(x+1) + (2-4x)}{x+1} &> 0 \\
	\frac{x^2 + x + 2-4x}{x+1} &> 0 \\
	\frac{x^2 -3x + 2}{x+1} &> 0 \\
	\label{eq9}
	\frac{(x-1)(x-2)}{x+1} &> 0
\end{align}

Since (\ref{eq9}) is undefined when $x = -1$,
we must have either $x < -1$ or $x > -1$.

Assume $x > -1$.
Then, (\ref{eq9}) holds if and only if
$(x - 1 > 0 \land  x-2 > 0) \lor (x-1 < 0 \land x-2 < 0)$.

Under this assumption, consider the following two cases:

\begin{itemize}
	\item
Assume $x - 1 > 0 \land x - 2 > 0$.
Then, $x > 1 \land x > 2 \implies x > 2$.

We conclude that a subset of our solution set is the interval $(2, \infty)$.

\item
Assume $x - 1 < 0 \land x - 2 < 0$.
Then, $x < 1 \land x < 2 \implies x < 1$.

We conclude that a subset of our solution set is the interval $(-1, 1)$.
\end{itemize}

Now, assume $x < -1$.
Then, (\ref{eq9}) holds if and only if
$(x - 1 > 0 \land  x-2 < 0) \lor (x-1 < 0 \land x-2 > 0)$.

Under this assumption, consider the following two cases:

\begin{itemize}
	\item
	Assume $x - 1 > 0 \land x - 2 < 0$.
	Then, $x > 1 \land x < 2$,
	but by assumption, $x < -1$, which is a contradiction,
	so we reject this assumption.
	\item
	Assume $x - 1 < 0 \land x - 2 > 0$.
	Then, $x < 1 \land x > 2$, which is a contradiction,
	so we reject this assumption.
\end{itemize}

Finally, we conclude that our complete solution set
must be the interval $(-1,1) \cup (2, \infty)$.

\medskip
\textbf{2:}

Let $A$ and $B$ be two sets such that
$A = \{x \in \reals: x > 0 \land x^2 \geq 3\}$
and $B = \{x \in \reals: x^2 < 3\} \cup \{x \in \reals: x \le 0\}$.

The definition of a Dedekind Cut was given in class as follows:
$(X,Y)$ forms a Dedekind Cut of $\reals$ if:
\begin{enumerate}
	\item
	$X \cup Y = \reals$
	\item
	$X \neq \emptyset \land Y \neq \emptyset$
	\item
	$(\forall x \in X)(\forall y \in Y)(x < y)$.
\end{enumerate}

\medskip
$(A,B)$ does not form a Dedekind Cut since $3 > 0$, and $3 \in A \land 0 \in B$,
violating the third constraint. Therefore, we instead consider the Dedekind cut $(B,A)$.

% By construction (axiom schema of comprehension), we have that $A \subseteq \reals \land B \subseteq \reals$.
% Since $0 \not\in A$, we must have $A \subset \reals$,
% and since $3^2 \geq 3 \implies 3 \not\in B$, we have $B \subset \reals$.
% Therefore the first constraint is satisfied.

Suppose $x \in \reals$. Then, either $x^2 < 3 \lor \neg (x^2 < 3) = x^2 \geq 3$.
If $x^2 < 3$, then $x \in B$. If $x^2 \geq 3$, then $x \in A$.
Since either the proposition or its negation is true about any $x \in \reals$,
we have demonstrated that $(\forall x \in \reals)(x \in A \lor x \in B)$,
therefore $A \cup B = \reals$ and the first constraint is satisfied.

Suppose $x = 0$. Then $x \in B$ by its definition, so $B \neq \emptyset$.
Suppose $x = 3$. Then, $x^2 = 9  \geq 3 \implies x \in A$, so $A \neq \emptyset$.
Therefore the second constraint is satisfied.

Let $a \in A$ and let $b \in B$.
By definition, we have that $a > 0$.
If $b \leq 0$, then $b \leq 0 < a \implies b < a$ is trivial.
Assume $b > 0$. By the definition of $B$, we have that $b^2 \leq 3$,
and by the definition of $A$, we have $a^2 \geq 3$.
Then by transitivity, we have $b^2 < 3 \le a^2 \implies b^2 < a^2$,
therefore $b^2 - a^2 = (b+a)(b-a) < 0$.
Since $b > 0 \land a > 0$, we must have $b + a > 0$,
and therefore since $(b+a)(b-a) < 0$ if and only if
$(b+a > 0 \land b-a < 0) \lor (b + a < 0 \land b - a > 0)$,
we must have that $b-a < 0$ and thus $b < a$.
We conclude that $(\forall b \in B)(\forall a \in A)(b < a)$,
therefore the third constraint is satisfied.

Since $(B,A)$ satisfies all of the constraints of the definition
of a Dedekind Cut of the reals, we conclude that it indeed must be a Dedekind Cut.

\end{document}
