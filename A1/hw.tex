\documentclass{article}

\usepackage{amsmath, amssymb}
\newcommand{\reals}{\mathbb{R}}

\title{Real Analysis Assignment 1}
\author{Joel Savitz}

\begin{document}

\maketitle

\textbf{1(a):}

Suppose we have the following equation:

\begin{align} \label{eq1}
	|x-1| + |x+2| = 0.5
\end{align}

We have $x \in \reals $ if and only if:
\begin{align} \label{eq0}
	x \le -2 \lor -2 \le x \le 1 \lor 1 \le x
\end{align}

since $(-\infty,-2] \cup [-2,1] \cup [1,\infty) = \reals$.

Assume $x \le -2$.
Then, $x-1 < 0 \implies |x-1| = -(x-1)$,
and $x+2 \le 0 \implies |x+2| = -(x+2)$,
so we can solve (\ref{eq1}) as follows:

\begin{align}
	& -(x-1) -(x+2) & = 0.5 \\
	= & -x+1 -x-2 & = 0.5 \\
	= & -2x-1 & = 0.5 \\
	\implies & -2x & = 1.5 \\
	\label{eq2} \implies & x & = \frac{1.5}{-2} = \frac{-3}{4}
\end{align}

However, we have by assumption that $x \le -2$,
and by (\ref{eq2}) we have that $x = \frac{-3}{4} > -2$,
which is a contradition.
Therefore, $x \le -2$ is false for any real solution to (\ref{eq1}).

Assume $-2 \le x \le 1$.
Then, $x-1 \le 0 \implies |x-1| = -(x-1)$,
and $x+2 \geq 0 \implies |x+2| = x+2$,
so we can solve (\ref{eq1}) as follows:

\begin{align}
	& -(x-1) + x+2 & = 0.5 \\
	= & -x+1 + x+2 & = 0.5 \\
	= & 3 & = 0.5
\end{align}

But of course, $3 = 0.5$ is absurd,
therefore we reach a contradiction
and conclude that $-2 \le x \le 1$ is false for any real solution to (\ref{eq1}).

Assume $x \geq 1$:
Then, $x-1 \geq 0 \implies |x-1| = x-1$,
and $x+2 > 0 \implies |x+2| = x+2$,
so we can solve (\ref{eq1}) as follows:

\begin{align}
	& x-1 +x+2 & = 0.5 \\
	= & 2x+1 & = 0.5 \\
	\implies & 2x & = -0.5 \\
	\label{eq3} \implies & x & = \frac{-0.5}{2} = \frac{-1}{4}
\end{align}

However, we have by assumption that $x \geq 1$,
and by (\ref{eq3}) we have that $x = \frac{-1}{4} < 1$,
which is a contradition.
Therefore, $x \geq 1$ is false for any real solution to (\ref{eq1}).

Finally, since we have $x \not \in (-\infty,-2] \land x \not\in [-2,1] \land x \not \in [1,\infty)$,
we have $x \not \in \reals = (-\infty,-2] \cup [-2,1] \cup [1,\infty)$,
therefore we have demonstrated that there are no real solutions to (\ref{eq1}).

\medskip
\textbf{1(b):}

Suppose we have the following equation:

\begin{align} \label{eq4}
	|x-1| + |x+2| = 3.5
\end{align}

We have $x \in \reals $ if and only if (\ref{eq0}) holds
since $(-\infty,-2] \cup [-2,1] \cup [1,\infty) = \reals$.

Assume $x \le -2$.
Then, $x-1 < 0 \implies |x-1| = -(x-1)$,
and $x+2 \le 0 \implies |x+2| = -(x+2)$,
so we can solve (\ref{eq4}) as follows:

\begin{align}
	& -(x-1) -(x+2) & = 3.5 \\
	= & -x+1 -x-2 & = 3.5 \\
	= & -2x-1 & = 3.5 \\
	\implies & -2x & = 4.5 \\
	\label{eq5} \implies & x & = \frac{4.5}{-2} = \frac{-9}{4}
\end{align}

Indeed, $x = \frac{-9}{4} \le -2$ so this solution is consistent
with our constaints and it must be a member of the solution set for (\ref{eq4}).

Assume $-2 \le x \le 1$.
Then, $x-1 \le 0 \implies |x-1| = -(x-1)$,
and $x+2 \geq 0 \implies |x+2| = x+2$,
so we can solve (\ref{eq4}) as follows:

\begin{align}
	& -(x-1) + x+2 & = 3.5 \\
	= & -x+1 + x+2 & = 3.5 \\
	= & 3 & = 3.5
\end{align}

But of course, $3 = 3.5$ is absurd,
therefore we reach a contradiction
and conclude that $-2 \le x \le 1$ is false for any real solution to (\ref{eq4}).

Assume $x \geq 1$:
Then, $x-1 \geq 0 \implies |x-1| = x-1$,
and $x+2 > 0 \implies |x+2| = x+2$,
so we can solve (\ref{eq4}) as follows:

\begin{align}
	& x-1 +x+2 & = 3.5 \\
	= & 2x+1 & = 3.5 \\
	\implies & 2x & = 2.5 \\
	\label{eq6} \implies & x & = \frac{2.5}{2} = \frac{5}{4}
\end{align}

Indeed, $x = \frac{5}{4} \geq 1$ so this solution is consistent
with our constaints and it must be a member of the solution set for (\ref{eq4}).

Finally, we can describe the real solutions
to (\ref{eq4}) by $x \in \{ \frac{-9}{4}, \frac{5}{4} \}$.

\end{document}
