\documentclass{article}

\usepackage{amsmath, amssymb, amsthm}
\newcommand{\reals}{\ensuremath{\mathbb{R}}}
\newcommand{\nats}{\ensuremath{\mathbb{N}}}
\newcommand{\eps}{\ensuremath{\epsilon}}
\newcommand{\neps}{\ensuremath{N_\epsilon}}
\newcommand{\overn}[1]{\ensuremath{\frac{#1}{n}}}
\newtheorem{clm}{Claim}

\title{Real Analysis Assignment 6}
\author{Joel Savitz}

\begin{document}

\maketitle

\textbf{Problem 1:}

\begin{clm}
	The minimal subsequential limit of some bounded infinite sequence $x_n$
	can be computed by the formula
	$\underset{k}{\sup}(\underset{n > k}{\inf} x_n)$
\end{clm}

\begin{proof}
	Since $x_n$ is bounded, it is bounded above.
	Denote the upper bound $N$, then $\forall n, x_n \le N$.
	Define a subsequence $y_k = \underset{n > k}{\inf} x_n$.
	Then, $y_1$ is the greatest lower bound of $\{ x_2, x_3, x_4, ... \}$.
	Since $\{ x_3, x_4, x_5, ... \}$ is a subset of $\{ x_2, x_3, x_4, ... \}$,
	we have that $y_1$ is also a lower bound of the latter set,
	but since $y_2 = \underset{n > k}{\inf} x_n$,
	we must have $y_1 \le y_2$.
	For some $y_k$,
	we have $y_{k-1}$ is a lower bound of $\{ x_{k + 1}, x_{k+2}, ...\}$,
	since $y_{k - 1} = \underset{n > k}{\inf} x_n$,
	but since $y_k = \underset{n > k}{\inf} x_n$,
	we must have $y_k \ge y_{k - 1}$.
	Thus $\forall k, y_k \le y_{k + 1}$, i.e. $y_k$ is monotone increasing.
	Then since $\forall n > k, y_k \le x_n$ and $\forall n, x_n \le N$,
	we have by transitivity that $y_k \le N$.
	Thus $y_k$ is bounded above.
	Since $y_k$ is a monotone increasing sequence bounded above,
	we have by the monotone convergence theorem that $\lim y_k$ exists,
	and it is equal to $\sup y_k$
	Let $A = \lim y_k$.
	Since $y_k$ has a limit $A$,
	every subsequence of $y_k$ has the same limit $A$.

	Now, pick some $n_1 \in \nats$.
	Then $y_{n_1} = \inf \{ x_{n_1}, x_{n_1 + 1}, x_{n_1 + 2}, ...\}$.
	Since $y_{n_1}$ is the greatest lower bound of
	$\{ x_{n_1}, x_{n_1 + 1}, x_{n_1 + 2}, ...\}$,
	we must have that $y_{n_1} + 1$ is not a lower bound of that set.
	Then, since $y_{n_1} + 1$ is not a lower bound,
	$\exists n_2 > n_1, y_{n_1} \le x_{n_2} < y_{n_1} + 1$.
	Continuing in the same way,
	we have $y_{n_2} = \inf \{ x_{n_2}, x_{n_2 + 1}, x_{n_2 + 2}, ...\}$,
	and since $y_{n_2}$ is the greatest lower bound of the set,
	we have that $y_{n_2} + \frac{1}{2}$ is not a lower bound of the set,
	thus $\exists n_3 > n_2, y_{n_2} \le x_{n_3} < y_{n_2} + \frac{1}{2}$.
	We can continue this indefinitely,
	such that for any $n_{k - 1}$,
	we have that $y_{n_{k - 1}} = \inf \{ x_{n_k}, x_{n_k + 1}, x_{n_k + 2}, ... \}$,
	and so $y_{n_{k - 1}} + \frac{1}{k - 1}$ is not a lower bound of the set,
	and thus $\exists n_k > n_{k - 1}, y_{n_{k - 1}} \le x_{n_k} < y_{n_{k - 1}} + \frac{1}{k - 1}$.
	Applying the squeze theorem and limit laws, we have:
	\begin{align}
		\lim (y_{n_{k - 1}}) \le & \lim x_{n_k} < & \lim (y_{n_{k - 1}} + \frac{1}{k - 1}) \\
		\lim (y_{n_{k - 1}}) \le & \lim x_{n_k} < & \lim (y_{n_{k - 1}}) + \lim(\frac{1}{k - 1}) \\
		A \le & \lim x_{n_k} \le & A + 0 \\
		\implies & \lim x_{n_k} = A
	\end{align}
	Thus $A$ is a subsequential limit of $x_n$.

	Consider some other subsequential limit of $x_n$,
	suppose $\lim x_{m_k} = B$.
	Then:
	\begin{align}
		y_{m_{k - 1}} = & \inf \{ x_{m_k}, x_{m_k + 1}, x_{m_k + 2}, ... \} \\
		\forall m_k, y_{m_k - 1} \le & x_{m_k} \\
		\lim y_{m_k - 1} \le & \lim x_{m_k} \\
		A \le & B 
	\end{align}
	Therefore, $A$ is the minimal subsequential limit of $x_n$,
	and it is the value of
	$\underset{k}{\sup}(\underset{n > k}{\inf} x_n)$.
\end{proof}

\textbf{Problem 2:}

\begin{clm}
	If $a_n$ and $b_n$ are Cauchy sequences,
	then $a_n b_n$ is a Cauchy sequence.
\end{clm}

\begin{proof}
	If a sequence is Cauchy, then it is bounded,
	and if a sequence is bounded,
	then it is bounded above.
	Thus, $\exists A, B \in \reals$,
	where $\forall n, a_n \le A$
	and $\forall n, b_n \le B$.
	By the definition of a Cauchy sequence,
	we have that
	\begin{align}
		\forall \eps > 0, \exists N_\eps, \forall n > N_\eps, m > N_\eps, |a_n - a_m | < \eps
	\end{align}
	and likewise
	\begin{align}
		\forall \eps > 0, \exists M_\eps, \forall n > M_\eps, m > M_\eps, |b_n - b_m | < \eps
	\end{align}
	Let $\eps > 0$.
	Thus we have the follwing:
	\begin{align}
	\exists N_{\frac{\eps}{2|B|}}, \forall n >  N_\frac{\eps}{2|B|}, m > N_\frac{\eps}{2|B|},
	|a_n - a_m| < \frac{\eps}{2|B|} \\
	\exists M_{\frac{\eps}{2|A|}}, \forall n >  M_\frac{\eps}{2|A|}, m > M_\frac{\eps}{2|A|},
	|b_n - b_m| < \frac{\eps}{2|A|}
	\end{align}
	Let $K_\eps = \max(N_{\frac{\eps}{2|B|}}, M_{\frac{\eps}{2|A|}})$.
	Then, $\forall n > K_\eps, m > K_\eps$, we have:
	\begin{align}
			& |B||a_n - a_m| < \frac{\eps}{2} \\
			& |A||b_n - b_m| < \frac{\eps}{2} \\
		\implies & |B||a_n - a_m| + |A||b_n - b_m| < \frac{\eps}{2} + \frac{\eps}{2} = \eps
	\end{align}
	Since $\forall n, a_n \le A$ and $\forall n, b_n \le B$,
	we have:
	\begin{align}
		|B||a_n - a_m| + |A||b_n + b_m| \geq & |b_m||a_n - a_m| + |a_n||b_n - b_m| \\
		|b_m||a_n - a_m| + |a_n||b_n - b_m| = & |a_n b_n - a_n b_m| + |a_n b_m - a_m b_m|
	\end{align}
	And by the triangle inequality:
	\begin{align}
		|a_n b_n - a_n b_m| + |a_n b_m - a_m b_m| \geq & |a_n b_n - a_n b_m + a_n b_m - a_m b_m| \\
		|a_n b_n - a_n b_m + a_n b_m - a_m b_m| = & |a_n b_n - a_m b_m|
	\end{align}
	Therefore, by transitivity, $|a_n b_n - a_m b_m| < \eps$.
	Thus $\forall \eps > 0, \exists K_\eps, \forall n > K_\eps, m > K_\eps, |a_n b_n - a_m b_m| < \eps$,
	and this is exactly the definition that $a_n b_n$ is a Cauchy sequence.
\end{proof}

\textbf{Problem 3:}

Define a sequence $x_n = \sqrt{n}$.

\medskip
\textbf{Part (a):}

\begin{clm}
	$\lim(x_{n+2} - x_n) = 0$
\end{clm}

\begin{proof}
	Let $\eps > 0$.
	By the Archimedian property,
	$\exists N_{\frac{4}{\eps^2}} \in \nats, N_{\frac{4}{\eps^2}} > \frac{4}{\eps^2}$.
	Thus, $\forall n > N_{\frac{4}{\eps^2}}, n > \frac{4}{\eps^2}$.
	Therefore:
	\begin{align}
		\frac{4}{n} < & \eps^2 \\
		\frac{2}{\sqrt{n}} < & \eps \\
		\frac{2}{\sqrt{2 + n} + \sqrt{n}} < & \frac{2}{\sqrt{n}} \\
		\frac{2}{\sqrt{2 + n} + \sqrt{n}} = & \frac{n + 2 - n}{\sqrt{2 + n} + \sqrt{n}}  \\
		= & \sqrt{n + 2} - \sqrt{n}
	\end{align}
	If we assume that for some natural $n$,
	$\sqrt{n + 2} \le \sqrt{n}$,
	we see that $n + 2 \le n$ and thus $2 \le 0$,
	which is absurd,
	therefore $\sqrt{n + 2} > \sqrt{n}$,
	and so $\sqrt{n + 2} - \sqrt{n} = |\sqrt{n + 2} - \sqrt{n}|$.
	By the transitivity of the above order relations,
	we then have $|\sqrt{n + 2} - \sqrt{n}| = 
	|\sqrt{n + 2} - \sqrt{n} - 0| < \eps$.
	Thus we have demonstrated
	that $\forall \eps > 0,
	\exists N_\eps,
	\forall n > N_\eps,
	|\sqrt{n + 2} - \sqrt{n} - 0| < \eps$,
	which is exactly the definition that
	$\lim(x_{n+2} - x_n) = 0$.
\end{proof}

\textbf{Part (b):}

\begin{clm}
	$x_n = \sqrt{n}$ is not a Cauchy sequence.
\end{clm}

\begin{proof}
	Let $\eps = 1$ and let $N$ be any natural number.
	Define $n = 4(N + 1)^2$ and $m = (N + 1)^2$.
	We see that:
	\begin{align}
		|x_n - x_m| = | \sqrt{n} - \sqrt{m}| = 	 &|\sqrt{4(N + 1)^2} - \sqrt{(N + 1)^2}| \\
							=& | 2(N + 1) - (N + 1)| \\
							=& | 2N + 2 - N - 1 | \\
							=& | N + 1 | \\
							=&  N + 1
	\end{align}
	Since $N \geq 1$, we have $N + 1 \geq 1 + 1 = 2 \geq 1 = \eps$.
	Thus we have demonstrated that
	$\exists \eps > 0, \forall N, \exists n > N, m > N, |x_n - x_m| \geq \eps$,
	which is the negation of the definition of a Cauchy sequence,
	therefore $x_n = \sqrt{n}$ is not a Cauchy sequence.
\end{proof}

\end{document}
