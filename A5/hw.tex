\documentclass{article}

\usepackage{amsmath, amssymb, amsthm}
\newcommand{\reals}{\ensuremath{\mathbb{R}}}
\newcommand{\nats}{\ensuremath{\mathbb{N}}}
\newcommand{\eps}{\ensuremath{\epsilon}}
\newcommand{\neps}{\ensuremath{N_\epsilon}}
\newcommand{\overn}[1]{\ensuremath{\frac{#1}{n}}}
\newcommand{\movern}{\overn{(-1)^n}}
\newtheorem{clm}{Claim}

\title{Real Analysis Assignment 5}
\author{Joel Savitz}

\begin{document}

\maketitle

\textbf{Problem 1:}

\begin{clm}
	Every monotone decreasing bounded below sequence has a limit
\end{clm}

\begin{proof}
	Let $x_n$ be a monotone decreasing sequence bounded below.
	Thus by definition, $\forall n, x_{n+1} \le x_n$.
	Let $M$ be the lower bound of $x_n$, i.e.e $\forall n, x_n \geq M$.
	Since $x_n$ is a sequence of real numbers bounded below,
	by the completeness of the reals we have that there exists
	a greatest lower bound, an infimum, of the set $\{ x_n : n \in \nats \}$.
	Let $a = \inf\{x_n : n \in \nats \}$.
	Let $\eps > 0$.
	Then, since $a$ is the greatest lower bound of $x_n$,
	there exists some $N$ such that $x_N < a + \eps$.
	Then, since $\forall n, x_{n+1} \le x_n$,
	we must have that $\forall n > N, x_n \le x_N$,
	and since $a$ is a lower bound of $x_n$,
	we have $\forall n, a \le x_n$, and since $\eps > 0$,
	we of course have $a - \eps < a$.
	Putting this all together,
	we have that $\forall n > N, a - \eps < a \le x_n \le x_N < a + \eps$,
	thus by transitivity, $\forall n > N, a - \eps < x_n < a + \eps$,
	therefore $-\eps < x_n - a < \eps \iff | x_n - a | < \eps$.
	We have demonstrated that
	$\forall \eps > 0, \exists N: \forall n > N, | x_n - a | < \eps$,
	which is exactly the definition that $\lim x_n = a$,
	therefore the limit of $x_n$ exists,
	and every monotone decreasing bounded belows sequence has a limit.
\end{proof}

\textbf{Problem 2:}


\begin{clm}
	The sequence $a_n = \begin{cases} a_1 = 2 \\ a_{n+1} = \frac{a_n^2 + 1}{2a_n} & n \geq 1 \end{cases}$ has a limit.
\end{clm}

\begin{proof}
	First, note that $a_1 = 2 \geq 1$.
	Then, let $t \geq 0$. Noting that $\forall x \in \reals, x^2 \geq 0$, we have the following:
	\begin{align}
		(\sqrt{t} - \frac{1}{\sqrt{t}})^2 & \geq 0 \\
		t - 2 + \frac{1}{t} & \geq 0 \\
		t + \frac{1}{t} & \geq 2
	\end{align}
	Thus $\forall t \geq 0, t + \frac{1}{t} \geq 2$.
	Knowing this, we can determine a bound for $a_{n+1}$ as follows:
	\begin{align}
		a_{n+1} = & \frac{a_n^2 + 1}{2a_n} \\
		= & a_n \frac{a_n + \frac{1}{a_n}}{2a_n} \\
		a_n\frac{2}{2a_n} \le & a_n \frac{a_n + \frac{1}{a_n}}{2a_n} \\
		1 \le & \frac{a_n^2 + 1}{2a_n} \\
		1 \le & a_{n+1}
	\end{align}
	Since $a_1 \geq 1 \land a_{n+1} \geq 1$,
	we have $\forall n, a_n \geq 1$,
	therefore $1$ is a lower bound of $a_n$
	and $a_n$ is bounded.
	Since $a_2 = \frac{a_1^2 + 1}{2a_1} = \frac{2^2 + 1}{2 \cdot 2} = \frac{5}{4}$,
	we see that $a_2 \le a_1$.
	Then, since $\forall n, a_n \geq 1 \implies \forall n, a_n^2 \geq 1$:
	\begin{align}
		a_{n+1} = & \frac{a_n^2 + 1}{2a_n} \\
		\frac{a_n^2 + a_n^2}{2a_n} \geq & \frac{a_n^2+1}{2a_n} \\
		\frac{2a_n^2}{2a_n} \geq & \frac{a_n^2+1}{2a_n} \\
		a_n \geq & a_{n+1}
	\end{align}
	Therefore, $a_n$ is monotone decreasing.
	Since $a_n$ is bounded below and monotone decreasing,
	By the monotone conversion theorem,
	the limit of $a_n$ exists.
\end{proof}

\underline{Computation of $\lim a_n$}

We know that $\lim a_n$ exists, so it makes sense to demote it by $a = \lim a_n$.

By the limit laws, $\lim a_{n+1} = \frac{\lim a_n^2 + 1}{2\lim{a_n}}$

Since they differ by finitely many terms, $\lim a_{n+1} = \lim a_n$,
so we can compute as follows:

\begin{align}
	a = & \frac{a^2 + 1}{2a} \\
	2a^2 = & a^2 + 1 \\
	a^2 = & 1 \\
	\implies & a \in \{ -1, 1\}
\end{align}

Suppose, for a contradiction, that $\lim a_n = -1$.
Then, $\forall \eps > 0 \exists N_\eps: \forall n > N_\eps, a_n \in (-1 - \eps, -1 + \eps)$,
therefore for any $\eps > 0$ this last membership relation must be true for at least one $a_n$.
Let $\eps = 1$. Then, some $a_n \in (-2, 0)$, but $\forall n, a_n \geq 1$,
which is a contradiction, thus $\lim a_n \neq -1$
and it must be the case that $\lim a_n = 1$.

\medskip
\textbf{Problem 3:}

\begin{clm}
	$0$ is not a sub-sequential limit of $x_n = (-1)^n - \frac{2}{n}$
\end{clm}

\begin{proof}
	Define the neighborhood $N = (\frac{-1}{2}, \frac{1}{2})$.
	Note that any $n$ is either even or odd,
	and if $n$ is even, then $(-1)^n = 1$,
	and if $n$ is odd, then $(-1)^n = -1$.
	Now let $n \in \nats$.
	$\forall n \in \nats, n > 0$ implies the following:
	\begin{align}
		& n > 0 \\
		\implies & \frac{1}{n} > 0 \\
		\implies & \frac{-1}{n} < 0 \\
		\implies & \frac{-2}{n} < 0 \\
		\implies & -1 - \frac{2}{n} < -1
	\end{align}
	If $n$ is odd, then then we have
	$(-1)^n - \frac{2}{n} < -1$.
	Thus for all odd $n$ we have  $x_n \not\in N$.
	Now let $n \geq 4$ and note the following:
	\begin{align}
		& n \geq 4 \\
		\implies & -n \le -4 \\
		\implies & n \le 2n-4 \\
		\implies & 1 \le 2\frac{n-2}{n} \\
		\implies & \frac{1}{2} \le \frac{n-2}{n} \\
		\implies & \frac{1}{2} \le 1 - \frac{2}{n}
	\end{align}
	If $n$ is even, then we have 
	$(-1)^n - \frac{2}{n} \geq \frac{1}{2}$,
	thus for any even $n \geq 4$, we have $x_n \not\in N$.
	Since $x_n \not\in N$ for any odd $n$,
	we must also have $x_n \not\in N$ for any odd $n > 4$,
	therefore $\forall n \geq 4, x_n \not\in (0- \frac{1}{2}, 0 + \frac{1}{2})$,
	and this demonstates that infinitely many $x_n$ fall outside a neighborhood of $0$,
	so we conclude that finitely many $x_n$ are in $N$.
	This is true if and only if $0$
	is not a subsequential limit of $x_n$.
\end{proof}

\end{document}
